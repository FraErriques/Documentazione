\documentclass{article}
\usepackage{amsmath}
\usepackage{amssymb}

\begin{document}
	
	\section*{Grassmann Formula in \(\mathbb{R}^4\) with Two Planes}
	
	Let \( U \) and \( V \) be two planes (i.e., 2-dimensional subspaces) in \( \mathbb{R}^4 \), both passing through the origin and not parallel. We apply the Grassmann formula:
	
	\[
	\dim(U + V) = \dim(U) + \dim(V) - \dim(U \cap V)
	\]
	
	Since both \( U \) and \( V \) are planes:
	\[
	\dim(U) = \dim(V) = 2
	\]
	
	Then:
	\[
	\dim(U + V) = 2 + 2 - \dim(U \cap V) = 4 - \dim(U \cap V)
	\]
	
	\subsection*{Possible Cases}
	
	\begin{itemize}
		\item \(\dim(U \cap V) = 0\): The planes intersect only at the origin. Then \(\dim(U + V) = 4\), and their span fills all of \(\mathbb{R}^4\).
		\item \(\dim(U \cap V) = 1\): The planes intersect along a line. Then \(\dim(U + V) = 3\), and their span is a 3D subspace of \(\mathbb{R}^4\).
		\item \(\dim(U \cap V) = 2\): The planes coincide. Then \(\dim(U + V) = 2\), which means they are the same plane.
	\end{itemize}
	
	Since the planes are not parallel, we exclude the third case.
	
	\subsection*{Example}
	
	Let:
	\[
	U = \text{span}\left\{ (1, 0, 0, 0),\ (0, 1, 0, 0) \right\}
	\]
	\[
	V = \text{span}\left\{ (0, 0, 1, 0),\ (0, 0, 0, 1) \right\}
	\]
	
	Then:
	\[
	U \cap V = \{0\}
	\]
	\[
	\dim(U + V) = 4
	\]
	
	These two planes intersect only at the origin and together span all of \(\mathbb{R}^4\).
	
\end{document}