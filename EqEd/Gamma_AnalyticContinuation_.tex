
\documentclass[11pt]{article}
\usepackage{amsmath,amssymb,amsthm}
\usepackage{amsmath}
\usepackage{amsfonts}   % defines \mathbb
% or, more commonly:
\usepackage{amssymb}    % loads amsfonts and more symbols
% and for good measure (recommended in math docs):
\usepackage{amsmath}    % better math environments
\usepackage{hyperref} % load last

\title{Analytic continuation of the Gamma function via the Hankel contour}
\author{}
\date{}

\begin{document}
	\maketitle
	
	\section{Euler's integral for \texorpdfstring{$\Gamma(z)$}{Gamma(z)}}
	
	For $\Re z > 0$, the Gamma function is defined by
	\begin{equation}
		\Gamma(z) = \int_0^{\infty} t^{z-1} e^{-t}\,dt.
	\end{equation}
	This integral converges only for $\Re z > 0$, so it does not directly define
	$\Gamma(z)$ on the whole complex plane.
	
	\section{The Hankel contour and the branch of the logarithm}
	
	We introduce a branch cut along the positive real axis and take the principal
	branch of the logarithm on $\mathbb{C} \setminus [0,\infty)$, so that
	\begin{equation}
		-\pi < \arg t < \pi.
	\end{equation}
	On this domain we define
	\begin{equation}
		t^{z-1} = e^{(z-1)\log t}.
	\end{equation}
	
	The \emph{Hankel contour} $\mathcal{H}$ is defined as follows:
	\begin{itemize}
		\item it starts at $+\infty$ just below the positive real axis,
		\item runs towards $0$ along this line,
		\item loops once counterclockwise around the origin along a small circle,
		\item and returns to $+\infty$ just above the positive real axis.
	\end{itemize}
	Thus $\mathcal{H}$ winds once around the origin and avoids the branch cut.
	
	\section{A contour integral involving \texorpdfstring{$\Gamma(z)$}{Gamma(z)}}
	
	Consider the contour integral
	\begin{equation}
		I(z) := \int_{\mathcal{H}} (-t)^{z-1} e^{-t}\,dt.
	\end{equation}
	We will relate $I(z)$ to $\Gamma(z)$ for $\Re z > 0$ and then use this
	representation to extend $\Gamma(z)$ analytically.
	
	On $\mathcal{H}$ we write
	\begin{equation}
		(-t)^{z-1} = e^{(z-1)\log(-t)},
	\end{equation}
	where $\log(-t)$ is defined using the chosen branch of the logarithm.
	
	\subsection{Contribution from the lower and upper rays}
	
	We parametrize the two rays of $\mathcal{H}$:
	
	\paragraph{Lower side.}
	On the lower side, $t = x$ with $x$ decreasing from $+\infty$ to $0$, just
	below the positive real axis. Then $dt = -dx$ and $-t = -x$ has argument
	$-\pi$, so
	\begin{equation}
		(-t)^{z-1} = x^{z-1} e^{-i\pi(z-1)}.
	\end{equation}
	Hence
	\begin{align}
		\int_{\text{lower}} (-t)^{z-1} e^{-t}\,dt
		&= \int_{\infty}^{0} x^{z-1} e^{-i\pi(z-1)} e^{-x} (-dx) \\
		&= e^{-i\pi(z-1)} \int_0^{\infty} x^{z-1} e^{-x}\,dx \\
		&= e^{-i\pi(z-1)} \Gamma(z),
	\end{align}
	valid for $\Re z > 0$.
	
	\paragraph{Upper side.}
	On the upper side, $t = x$ with $x$ increasing from $0$ to $\infty$, just
	above the positive real axis. Then $dt = dx$ and $-t = -x$ has argument
	$+\pi$, so
	\begin{equation}
		(-t)^{z-1} = x^{z-1} e^{+i\pi(z-1)}.
	\end{equation}
	Thus
	\begin{align}
		\int_{\text{upper}} (-t)^{z-1} e^{-t}\,dt
		&= \int_0^{\infty} x^{z-1} e^{+i\pi(z-1)} e^{-x}\,dx \\
		&= e^{+i\pi(z-1)} \Gamma(z).
	\end{align}
	
	\paragraph{Small circle around the origin.}
	The small circular arc around $0$ contributes $0$ in the limit (for
	$\Re z > 0$), since $t^{z-1}$ is integrable near $0$ and the arc length
	tends to $0$.
	
	\subsection{Expression for \texorpdfstring{$I(z)$}{I(z)}}
	
	Summing the contributions from the lower and upper rays, we obtain
	\begin{align}
		I(z)
		&= \left( e^{-i\pi(z-1)} + e^{+i\pi(z-1)} \right) \Gamma(z) \\
		&= 2\cos\big(\pi(z-1)\big)\,\Gamma(z).
	\end{align}
	Using the identity
	\begin{equation}
		\cos\big(\pi(z-1)\big) = -\cos(\pi z)
	\end{equation}
	and the relation
	\begin{equation}
		\sin(\pi z) = \frac{e^{i\pi z} - e^{-i\pi z}}{2i},
	\end{equation}
	one can rewrite $I(z)$ in terms of $\sin(\pi z)$. A standard and convenient
	form of the Hankel representation is
	\begin{equation}
		\Gamma(z) = \frac{1}{2i\sin(\pi z)} \int_{\mathcal{H}} (-t)^{z-1} e^{-t}\,dt,
	\end{equation}
	valid initially for $\Re z > 0$ and $z \notin \mathbb{Z}$ (to avoid the zeros
	of $\sin(\pi z)$).
	
	\section{Analytic continuation}
	
	Define
	\begin{equation}
		F(z) := \frac{1}{2i\sin(\pi z)} \int_{\mathcal{H}} (-t)^{z-1} e^{-t}\,dt.
	\end{equation}
	We now analyze $F(z)$ as a function of $z$.
	
	\subsection{Analyticity of the integral}
	
	For each fixed $t$ on $\mathcal{H}$, the integrand $(-t)^{z-1} e^{-t}$ is
	entire in $z$. The contour $\mathcal{H}$ is fixed and avoids the singularity
	at $t=0$ by looping around it. The factor $e^{-t}$ ensures convergence at
	infinity for all $z \in \mathbb{C}$. Standard arguments (dominated convergence,
	Morera's theorem) show that the map
	\begin{equation}
		z \mapsto \int_{\mathcal{H}} (-t)^{z-1} e^{-t}\,dt
	\end{equation}
	is an entire function of $z$.
	
	\subsection{Poles and agreement with \texorpdfstring{$\Gamma(z)$}{Gamma(z)}}
	
	The only possible singularities of $F(z)$ come from the factor
	$1/\sin(\pi z)$, which has simple poles at $z \in \mathbb{Z}$. The integral
	itself is entire, so $F(z)$ has at most simple poles at the integers.
	
	For $\Re z > 0$, we have already computed $I(z)$ and found
	\begin{equation}
		I(z) = 2i\sin(\pi z)\,\Gamma(z),
	\end{equation}
	up to the standard trigonometric identities. Hence, on the half-plane
	$\Re z > 0$,
	\begin{equation}
		F(z) = \Gamma(z).
	\end{equation}
	Thus $F(z)$ extends $\Gamma(z)$ analytically from $\Re z > 0$ to
	$\mathbb{C} \setminus \mathbb{Z}$.
	
	A more detailed analysis shows that the poles at positive integers are
	removable (the integral vanishes appropriately there), while the poles at
	$0,-1,-2,\dots$ remain, matching the known singularities of $\Gamma(z)$.
	Therefore $F(z)$ is the analytic continuation of $\Gamma(z)$ to
	$\mathbb{C} \setminus \{0,-1,-2,\dots\}$.
	
	\section{Final Hankel representation}
	
	We conclude that the Gamma function admits the Hankel contour representation
	\begin{equation}
		\boxed{
			\Gamma(z) = \frac{1}{2i\sin(\pi z)}
			\int_{\mathcal{H}} (-t)^{z-1} e^{-t}\,dt,
			\quad z \in \mathbb{C} \setminus \{0,-1,-2,\dots\}.
		}
	\end{equation}
	This formula provides an explicit analytic continuation of $\Gamma(z)$ from
	the half-plane $\Re z > 0$ to the whole complex plane minus its simple poles
	at the nonpositive integers.
	
\end{document}