 \documentclass{article}
 \usepackage{amsmath}
 \usepackage{amsfonts}
 \usepackage{amssymb}
 
 \begin{document}
 	
 	\section*{Exterior Derivative of a Non-Exact 1-Form in \(\mathbb{R}^2\)}
 	
 	Let us consider the 1-form:
 	\[
 	\omega = -y\,dx + x\,dy
 	\]
 	
 	This 1-form is \textbf{not exact} on all of \(\mathbb{R}^2\), though it is closed on \(\mathbb{R}^2 \setminus \{(0,0)\}\). We now compute its exterior derivative.
 	
 	\subsection*{Compute the Exterior Derivative}
 	
 	Apply the exterior derivative:
 	\[
 	d\omega = d(-y\,dx + x\,dy)
 	\]
 	
 	Break it into parts:
 	\[
 	d(-y\,dx) = -dy \wedge dx, \quad d(x\,dy) = dx \wedge dy
 	\]
 	
 	So:
 	\[
 	d\omega = -dy \wedge dx + dx \wedge dy
 	\]
 	
 	Using the antisymmetry of the wedge product:
 	\[
 	dy \wedge dx = -dx \wedge dy
 	\]
 	
 	Therefore:
 	\[
 	d\omega = -(-dx \wedge dy) + dx \wedge dy = dx \wedge dy + dx \wedge dy = 2\,dx \wedge dy
 	\]
 	
 	\subsection*{Final Result}
 	
 	The exterior derivative of the 1-form \( \omega = -y\,dx + x\,dy \) is:
 	\[
 	d\omega = 2\,dx \wedge dy
 	\]
 	
 	This is a \textbf{nonzero 2-form}, indicating that \( \omega \) is not closed and hence not exact on all of \(\mathbb{R}^2\).
 	
 \end{document}