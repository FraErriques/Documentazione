\documentclass{article}
\usepackage{amsmath}
\usepackage{amssymb}
\usepackage{amsfonts}

\begin{document}
	
	\section*{Why the 1-Form \(\omega = -y\,dx + x\,dy\) Is Not Exact at the Origin}
	
	The 1-form
	\[
	\omega = -y\,dx + x\,dy
	\]
	is perfectly well-defined and smooth at the origin \((0,0)\). However, the subtlety arises when we ask whether this 1-form is \textbf{exact}, i.e., whether there exists a scalar function \( f(x, y) \) such that \( df = \omega \).
	
	\subsection*{Topology and Exactness}
	
	In differential geometry, a key result is:
	
	\textbf{Every closed 1-form on a simply connected domain is exact.}
	
	Let us unpack this:
	
	\begin{itemize}
		\item A 1-form \( \omega \) is \textbf{closed} if \( d\omega = 0 \).
		\item A 1-form is \textbf{exact} if \( \omega = df \) for some scalar function \( f \).
		\item A domain is \textbf{simply connected} if every loop can be continuously contracted to a point (i.e., the domain has no holes).
	\end{itemize}
	
	Now, for our 1-form:
	\[
	\omega = -y\,dx + x\,dy
	\]
	we computed:
	\[
	d\omega = 2\,dx \wedge dy \neq 0
	\]
	So \(\omega\) is not even closed, and therefore not exact.
	
	\subsection*{A More Subtle Example: The Normalized 1-Form}
	
	Consider the normalized version:
	\[
	\tilde{\omega} = \frac{-y\,dx + x\,dy}{x^2 + y^2}
	\]
	
	This 1-form is \textbf{closed} (you can verify that \( d\tilde{\omega} = 0 \)), but it is \textbf{not exact} on all of \( \mathbb{R}^2 \setminus \{(0,0)\} \). Why?
	
	Because the domain \( \mathbb{R}^2 \setminus \{(0,0)\} \) is \textbf{not simply connected}—it has a ``hole'' at the origin. This hole prevents us from defining a global potential function \( f \) such that \( df = \tilde{\omega} \).
	
	\subsection*{Summary}
	
	\begin{itemize}
		\item The original 1-form \( \omega = -y\,dx + x\,dy \) is defined everywhere, including the origin.
		\item But when we talk about \textbf{exactness}, the topology of the domain matters.
		\item The normalized version \( \tilde{\omega} = \frac{-y\,dx + x\,dy}{x^2 + y^2} \) is closed but not exact on \( \mathbb{R}^2 \setminus \{(0,0)\} \) because the domain is not simply connected.
		\item This is a classic example in differential geometry illustrating how \textbf{topology controls the behavior of differential forms}.
	\end{itemize}
	
\end{document}