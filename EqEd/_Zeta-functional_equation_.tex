
\documentclass{article}
\usepackage{amsmath}

\title{Riemann Zeta Functional Equation}
\author{Francesco}
\date{\today}

\begin{document}
	
	\section*{How Analytic Continuation Enters the Functional Equation of the Riemann Zeta Function}
	
	The Riemann zeta function is originally defined by the Dirichlet series
	\[
	\zeta(s) = \sum_{n=1}^{\infty} \frac{1}{n^s},
	\qquad \Re(s) > 1.
	\]
	The functional equation relates $\zeta(s)$ to $\zeta(1-s)$, but since
	\[
	\Re(s) > 1 \quad \Rightarrow \quad \Re(1-s) < 0,
	\]
	the Dirichlet series cannot be used to evaluate $\zeta(1-s)$.
	Therefore \emph{analytic continuation is needed before the functional equation can even make sense}.
	
	\subsection*{1. Mellin transform representation (valid only for $\Re(s)>1$)}
	
	A classical identity gives
	\[
	\zeta(s)\Gamma(s)
	= \int_0^\infty \frac{x^{s-1}}{e^{x} - 1}\, dx,
	\qquad \Re(s) > 1.
	\]
	This integral representation still has the same limitation: it converges only in the half-plane $\Re(s)>1$.
	
	\subsection*{2. Splitting the integral: beginning analytic continuation}
	
	We split the integral:
	\[
	\int_0^\infty = \int_0^1 + \int_1^\infty.
	\]
	The second part,
	\[
	\int_1^\infty \frac{x^{s-1}}{e^x - 1}\, dx,
	\]
	converges for all $s$ and defines an \emph{entire function}.
	
	The behaviour near $x=0$ is responsible for divergence. The expansion
	\[
	\frac{1}{e^x-1}
	= \frac{1}{x} - \frac{1}{2} + \frac{x}{12} - \frac{x^3}{720} + \cdots
	\]
	allows us to subtract finitely many terms:
	\[
	\frac{1}{e^x-1} - \left(
	\frac{1}{x} - \frac{1}{2}
	\right),
	\]
	which removes the singularity at $x=0$.
	
	The modified integrand is regular at $x=0$, so
	\[
	\int_0^1 \left[\frac{x^{s-1}}{e^x-1} - x^{s-2} + \frac{1}{2}x^{s-1} \right] dx
	\]
	converges for all complex $s$.
	This produces an analytic continuation of $\zeta(s)\Gamma(s)$ to the whole complex plane.
	
	\subsection*{3. Introducing the theta function}
	
	Now consider the theta function
	\[
	\theta(t) = 1 + 2 \sum_{n=1}^\infty e^{-\pi n^2 t},
	\]
	which satisfies the modular identity
	\[
	\theta(t) = t^{-1/2}\,\theta(1/t).
	\]
	
	Using the identity
	\[
	\int_0^\infty x^{s/2 - 1} \bigl(\theta(x) - 1\bigr)\, dx
	= \pi^{-s/2}\Gamma\!\left(\frac{s}{2}\right)\zeta(s),
	\]
	and applying the modular transformation, one obtains a relation between $\zeta(s)$ and $\zeta(1-s)$.
	
	The key point: these theta-function integrals are valid for \emph{all} $s$.
	This is why analytic continuation was needed—so that the $\zeta(s)$ which appears in this identity is well-defined outside $\Re(s)>1$.
	
	\subsection*{4. The functional equation emerges}
	
	One finally obtains the identity
	\[
	\pi^{-s/2}\Gamma\!\left(\frac{s}{2}\right)\zeta(s)
	=
	\pi^{-(1-s)/2}\Gamma\!\left(\frac{1-s}{2}\right)\zeta(1-s).
	\]
	
	Equivalently,
	\[
	\boxed{
		\zeta(s)
		= 2^s \pi^{s-1}\sin\!\left(\frac{\pi s}{2}\right)
		\Gamma(1-s)\, \zeta(1-s)
	}.
	\]
	
	\subsection*{Summary: where analytic continuation appears}
	
	\begin{itemize}
		\item The Dirichlet series only defines $\zeta(s)$ for $\Re(s)>1$.
		\item To write $\zeta(1-s)$ in the functional equation, we need values outside this region.
		\item Analytic continuation is achieved by rewriting the integral representation so that the $x=0$ singularity is removed.
		\item Once $\zeta(s)$ is defined for all $s$, the modular transformation of the theta function produces a natural symmetry between $s$ and $1-s$.
	\end{itemize}
	
	Thus \emph{analytic continuation is a necessary prerequisite} for the functional equation to hold.
	
	
 \end{document}
