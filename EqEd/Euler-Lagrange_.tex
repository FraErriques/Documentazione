\documentclass{article}
\usepackage{amsmath}

\title{Sample LaTeX Document}
\author{Your Name}
\date{\today}

\begin{document}

Absolutely! Let's derive Newton's second law \( F = ma \) from the Euler-Lagrange equation step by step.

\begin{equation}

### Euler-Lagrange Equation
The Euler-Lagrange equation is given by:
$$ \frac{d}{dt} \left( \frac{\partial L}{\partial \dot{x}} \right) - \frac{\partial L}{\partial x} = 0 $$
where \( L \) is the Lagrangian, \( x \) is the position, and \( \dot{x} \) is the velocity.

### Lagrangian for a Particle
For a particle of mass \( m \) moving in a potential \( V(x) \), the Lagrangian \( L \) is:
$$ L = T - V = \frac{1}{2} m \dot{x}^2 - V(x) $$

### Applying the Euler-Lagrange Equation
1. **Calculate \( \frac{\partial L}{\partial \dot{x}} \):**
   $$ \frac{\partial L}{\partial \dot{x}} = \frac{\partial}{\partial \dot{x}} \left( \frac{1}{2} m \dot{x}^2 - V(x) \right) = m \dot{x} $$

2. **Calculate \( \frac{d}{dt} \left( \frac{\partial L}{\partial \dot{x}} \right):**
   $$ \frac{d}{dt} \left( \frac{\partial L}{\partial \dot{x}} \right) = \frac{d}{dt} (m \dot{x}) = m \ddot{x} $$

3. **Calculate \( \frac{\partial L}{\partial x} \):**
   $$ \frac{\partial L}{\partial x} = \frac{\partial}{\partial x} \left( \frac{1}{2} m \dot{x}^2 - V(x) \right) = - \frac{\partial V}{\partial x} $$

### Substitute into the Euler-Lagrange Equation
Substitute these results into the Euler-Lagrange equation:
$$ m \ddot{x} - \left( - \frac{\partial V}{\partial x} \right) = 0 $$
which simplifies to:
$$ m \ddot{x} = \frac{\partial V}{\partial x} $$

### Newton's Second Law
Recognize that \( \frac{\partial V}{\partial x} \) is the force \( F \) acting on the particle:
$$ F = - \frac{\partial V}{\partial x} $$

Thus, we have:
$$ F = m \ddot{x} $$
or equivalently:
$$ F = ma $$

And there you have it! Newton's second law \( F = ma \) derived from the Euler-Lagrange equation. If you have any questions or need further clarification, feel free to ask!

\end{equation}

\end{document}
