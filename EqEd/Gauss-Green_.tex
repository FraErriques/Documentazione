
\documentclass{article}
\usepackage{amsmath}

\title{Sample LaTeX Document}
\author{Your Name}
\date{\today}

\begin{document}

\maketitle

\section{Question to AI}
Good morning. Could You please show me the Green Gauss integration formula for a scalar field in the plane (x,y), f:(x,y)->z , f:R2->R, on a square chain [a,b]x[c,d] , f is C1 on R2


\section{Answer from AI}
%Good morning! Certainly, here is the Green-Gauss integration formula for the scalar field \( f(x, y) \) 
%defined in \( \mathbb{R}^2 \) and \( C^1 \) on \( \mathbb{R}^2 \), applied to the square domain \([a,b] \times [c,d]\):

\[
\int_{a}^{b}\int_{c}^{d} \left( \frac{\partial P}{\partial x} + \frac{\partial Q}{\partial y} \right) dx \, dy = \int_{\partial R} \big( P \, dx + Q \, dy \big)
\]

Where:
- \( R = [a,b] \times [c,d] \) is the square domain.
- \( \partial R \) is the closed boundary of \( R \), oriented positively (counterclockwise).
- \( P(x, y) \) and \( Q(x, y) \) are components of a vector field \( \mathbf{F} = (P, Q) \).

This formula essentially relates the integral of the divergence of \( \mathbf{F} \) over \( R \) to the line integral of \( \mathbf{F} \) along the boundary \( \partial R \). In your case, the scalar field \( f(x,y) \) can be substituted appropriately into the formula to analyze its behavior on the given domain.

Let me know if you’d like further clarification or examples!

\end{document}
