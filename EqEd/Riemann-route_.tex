\documentclass{article}
\usepackage{amsmath}
\begin{document}

\section*{Expansion of \((-z)^{s-1}\)}

We begin with the definition of complex exponentiation:

\[
(-z)^{s-1} = e^{(s-1) \log(-z)}
\]

Using the principal branch of the complex logarithm:

\[
\log(-z) = \log z + i\pi
\]

Substitute into the exponent:

\[
(-z)^{s-1} = e^{(s-1)(\log z + i\pi)} = e^{(s-1)\log z} \cdot e^{i\pi(s-1)}
\]

Since \( e^{(s-1)\log z} = z^{s-1} \), we obtain:

\[
(-z)^{s-1} = z^{s-1} \cdot e^{i\pi(s-1)}
\]

This identity shows that raising a negative complex number to a power introduces a phase shift of \( \pi(s-1) \), which connects to trigonometric functions via Euler's formula:

\[
e^{i\pi(s-1)} = \cos(\pi(s-1)) + i \sin(\pi(s-1))
\]

\end{document}

