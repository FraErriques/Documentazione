\documentclass{article}
\usepackage{amsmath}
\usepackage{amsfonts}
\usepackage{geometry}
\geometry{margin=1in}

\title{The Minkowski Metric in Special Relativity}
\author{}
\date{}

\begin{document}
	
	\maketitle
	
	\section*{What Is Minkowski Space?}
	
	Minkowski space is a four-dimensional spacetime combining three spatial dimensions $(x, y, z)$ and one time dimension $(t)$. It is named after Hermann Minkowski, who reformulated Einstein’s special relativity using geometric principles.
	
	Unlike Euclidean space, Minkowski space uses a pseudo-Euclidean metric that treats time differently from space.
	
	\section*{The Minkowski Metric}
	
	The metric defines how distances (or intervals) are measured between two events in spacetime. In Minkowski space, the spacetime interval $s^2$ between two events is given by:
	
	\[
	s^2 = -c^2 t^2 + x^2 + y^2 + z^2
	\]
	
	Or, in compact form using the metric tensor $\eta_{\mu\nu}$:
	
	\[
	s^2 = \eta_{\mu\nu} x^\mu x^\nu
	\]
	
	Where:
	\begin{itemize}
		\item $x^\mu$ are the coordinates, with $x^0 = ct$, $x^1 = x$, $x^2 = y$, $x^3 = z$
		\item $\eta_{\mu\nu}$ is the Minkowski metric tensor, typically written as:
	\end{itemize}
	
	\[
	\eta_{\mu\nu} = \begin{pmatrix}
		-1 & 0 & 0 & 0 \\
		0 & 1 & 0 & 0 \\
		0 & 0 & 1 & 0 \\
		0 & 0 & 0 & 1
	\end{pmatrix}
	\]
	
	This signature $(-+++)$ is common, though some conventions use $(+---)$ instead.
	
	\section*{Why Is This Important?}
	
	\begin{itemize}
		\item \textbf{Invariant Interval}: The spacetime interval $s^2$ is invariant under Lorentz transformations. All observers, regardless of their motion, agree on this quantity.
		\item \textbf{Causal Structure}: Depending on the sign of $s^2$, events can be:
		\begin{itemize}
			\item \textbf{Timelike} ($s^2 < 0$): One event can causally influence the other.
			\item \textbf{Spacelike} ($s^2 > 0$): No causal connection is possible.
			\item \textbf{Lightlike} ($s^2 = 0$): Events are connected by a light signal.
		\end{itemize}
	\end{itemize}
	
	\section*{A Quick Example}
	
	Consider two events:
	\begin{itemize}
		\item Event A: $(t = 0, x = 0, y = 0, z = 0)$
		\item Event B: $(t = 1\,\text{s}, x = 3 \times 10^8\,\text{m}, y = 0, z = 0)$
	\end{itemize}
	
	Then:
	
	\[
	s^2 = -c^2(1)^2 + (3 \times 10^8)^2 = 0
	\]
	
	This is a \textbf{lightlike interval}—a photon could travel from A to B.\\
	
Follows an example of Lorentz invariance:\\
\\
\\
\title{ Invariance of the Spacetime Interval under Lorentz Transformations}

	\maketitle
	
	\section*{Goal}
	
	We aim to show that the spacetime interval between two events,
	\[
	s^2 = -c^2 t^2 + x^2 + y^2 + z^2,
	\]
	remains unchanged under a Lorentz transformation.
	
	\section*{Step-by-Step Sketch}
	
	\subsection*{1. Define the Interval in 4D Spacetime}
	
	Let the coordinates of an event be represented by a four-vector:
	\[
	x^\mu = (ct, x, y, z).
	\]
	
	The spacetime interval is given by:
	\[
	s^2 = \eta_{\mu\nu} x^\mu x^\nu,
	\]
	where \( \eta_{\mu\nu} \) is the Minkowski metric tensor:
	\[
	\eta_{\mu\nu} = \begin{pmatrix}
		-1 & 0 & 0 & 0 \\
		0 & 1 & 0 & 0 \\
		0 & 0 & 1 & 0 \\
		0 & 0 & 0 & 1
	\end{pmatrix}.
	\]
	
	\subsection*{2. Apply a Lorentz Transformation}
	
	A Lorentz transformation relates coordinates in one inertial frame to another:
	\[
	x'^\mu = \Lambda^\mu_{\ \nu} x^\nu,
	\]
	where \( \Lambda^\mu_{\ \nu} \) is the Lorentz transformation matrix.
	
	\subsection*{3. Compute the Interval in the New Frame}
	
	The interval in the primed frame is:
	\[
	s'^2 = \eta_{\mu\nu} x'^\mu x'^\nu = \eta_{\mu\nu} \Lambda^\mu_{\ \alpha} x^\alpha \Lambda^\nu_{\ \beta} x^\beta.
	\]
	
	Rewriting:
	\[
	s'^2 = x^\alpha x^\beta \left( \Lambda^\mu_{\ \alpha} \eta_{\mu\nu} \Lambda^\nu_{\ \beta} \right).
	\]
	
	\subsection*{4. Use Lorentz Invariance Condition}
	
	Lorentz transformations preserve the Minkowski metric:
	\[
	\Lambda^\mu_{\ \alpha} \eta_{\mu\nu} \Lambda^\nu_{\ \beta} = \eta_{\alpha\beta}.
	\]
	
	Therefore:
	\[
	s'^2 = x^\alpha x^\beta \eta_{\alpha\beta} = s^2.
	\]
	
	\section*{Conclusion}
	
	Since \( s'^2 = s^2 \), the spacetime interval is invariant under Lorentz transformations. This means all inertial observers agree on the value of \( s^2 \), regardless of their relative motion.\\
\\
\\
	
Follows a worked out example of a Lorentz-boost along the x-axis\\
\\
\\
 
\title{Lorentz Boost and Invariance of the Spacetime Interval}
\author{}
\date{}


	
	\maketitle
	
	\section*{Setup}
	
	Consider two events in spacetime:
	\begin{itemize}
		\item Event A: $(t_1, x_1)$
		\item Event B: $(t_2, x_2)$
	\end{itemize}
	
	We ignore the $y$ and $z$ coordinates since the boost is along the $x$-axis.
	
	The spacetime interval between these two events is:
	\[
	s^2 = -c^2(t_2 - t_1)^2 + (x_2 - x_1)^2
	\]
	
	Define:
	\[
	\Delta t = t_2 - t_1, \quad \Delta x = x_2 - x_1
	\]
	
	So:
	\[
	s^2 = -c^2 \Delta t^2 + \Delta x^2
	\]
	
	\section*{Lorentz Transformation Along the x-axis}
	
	A Lorentz boost along the $x$-axis with velocity $v$ transforms coordinates as:
	\[
	\begin{aligned}
		x' &= \gamma(x - vt) \\
		t' &= \gamma\left(t - \frac{v}{c^2}x\right)
	\end{aligned}
	\]
	where:
	\[
	\gamma = \frac{1}{\sqrt{1 - \frac{v^2}{c^2}}}
	\]
	
	Apply the transformation to the differences:
	\[
	\begin{aligned}
		\Delta x' &= \gamma(\Delta x - v \Delta t) \\
		\Delta t' &= \gamma\left(\Delta t - \frac{v}{c^2} \Delta x\right)
	\end{aligned}
	\]
	
	\section*{Compute the Interval in the Primed Frame}
	
	\[
	\begin{aligned}
		s'^2 &= -c^2 (\Delta t')^2 + (\Delta x')^2 \\
		&= -c^2 \gamma^2 \left(\Delta t - \frac{v}{c^2} \Delta x\right)^2 + \gamma^2 (\Delta x - v \Delta t)^2
	\end{aligned}
	\]
	
	Expand both terms:
	
	\textbf{Time term:}
	\[
	-c^2 \gamma^2 \left(\Delta t^2 - 2\frac{v}{c^2} \Delta t \Delta x + \frac{v^2}{c^4} \Delta x^2\right)
	= -\gamma^2 \left(c^2 \Delta t^2 - 2v \Delta t \Delta x + \frac{v^2}{c^2} \Delta x^2\right)
	\]
	
	\textbf{Space term:}
	\[
	\gamma^2 \left(\Delta x^2 - 2v \Delta x \Delta t + v^2 \Delta t^2\right)
	\]
	
	Add both:
	\[
	s'^2 = \gamma^2 \left[ -c^2 \Delta t^2 + \Delta x^2 + v^2 \Delta t^2 - \frac{v^2}{c^2} \Delta x^2 \right]
	\]
	
	Group terms:
	\[
	s'^2 = \gamma^2 \left[ \Delta x^2 \left(1 - \frac{v^2}{c^2}\right) - \Delta t^2 \left(c^2 - v^2\right) \right]
	\]
	
	Factor out:
	\[
	s'^2 = \gamma^2 \left(1 - \frac{v^2}{c^2}\right) \left[ \Delta x^2 - c^2 \Delta t^2 \right]
	\]
	
	But:
	\[
	\gamma^2 \left(1 - \frac{v^2}{c^2}\right) = 1
	\]
	
	So:
	\[
	s'^2 = \Delta x^2 - c^2 \Delta t^2 = s^2
	\]
	
	\section*{Conclusion}
	
	We have shown that:
	\[
	s'^2 = s^2
	\]
	Therefore, the spacetime interval is invariant under a Lorentz boost along the $x$-axis.
	
 	
\end{document}
