
\documentclass{article}
\usepackage{amsmath, amssymb}
\usepackage{geometry}
\geometry{margin=1in}

\title{Divergence as the Exterior Derivative of the Rotor via Differential Forms}
\author{}
\date{}

\begin{document}

\maketitle

\section*{Introduction}

In differential geometry, vector calculus operations such as gradient, curl (rotor), and divergence can be elegantly expressed using differential forms and the exterior derivative. This document explains how the divergence of a vector field in $\mathbb{R}^3$ can be obtained from the exterior derivative of the rotor, using the Hodge star operator.

\section*{Vector Field and Associated Differential Form}

Let $\mathbf{F}$ be a vector field in $\mathbb{R}^3$:
\[
\mathbf{F} = F_x \, \mathbf{i} + F_y \, \mathbf{j} + F_z \, \mathbf{k}
\]

We associate to $\mathbf{F}$ a 1-form:
\[
\omega = F_x \, dx + F_y \, dy + F_z \, dz
\]

\section*{Exterior Derivative: Curl as a 2-Form}

Taking the exterior derivative of $\omega$ yields a 2-form:
\[
d\omega = \left( \frac{\partial F_y}{\partial x} - \frac{\partial F_x}{\partial y} \right) dx \wedge dy + \left( \frac{\partial F_z}{\partial y} - \frac{\partial F_y}{\partial z} \right) dy \wedge dz + \left( \frac{\partial F_x}{\partial z} - \frac{\partial F_z}{\partial x} \right) dz \wedge dx
\]

This 2-form corresponds to the curl (rotor) of $\mathbf{F}$.

\section*{Exterior Derivative of a 2-Form}

Applying the exterior derivative again:
\[
d(d\omega) = 0
\]

This follows from the property of the exterior derivative:
\[
d^2 = 0
\]

Hence, we cannot directly obtain divergence by applying $d$ again to $d\omega$.

\section*{Using the Hodge Star Operator}

To obtain the divergence, we use the Hodge star operator $\star$, which maps $k$-forms to $(n-k)$-forms in an $n$-dimensional space.

Let $\eta = d\omega$ be the 2-form representing the curl. Then:
\[
\star \eta \quad \text{is a 1-form}
\]

Taking the exterior derivative:
\[
d(\star \eta)
\]

This is a 2-form, and applying the Hodge star again:
\[
\star d \star \omega
\]

This expression corresponds to the divergence of $\mathbf{F}$:
\[
\text{div}(\mathbf{F}) = \star d \star \omega
\]

\section*{Conclusion}

Using differential forms and the Hodge star operator, we can express the divergence of a vector field as:
\[
\text{div}(\mathbf{F}) = \star d \star \omega
\]

This formulation provides a powerful and coordinate-free way to understand classical vector calculus operations.

\end{document}
